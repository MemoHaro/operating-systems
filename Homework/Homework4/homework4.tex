\documentclass{article}

\usepackage[utf8]{inputenc}
\usepackage[top=1.5in,left=1in,right=1in,bottom=1in,headheight=1in]{geometry}
\usepackage{fancyhdr}
\usepackage{multicol}
\usepackage{alltt}
\usepackage{multirow}

\pagestyle{fancy}

\rhead{
  Stefan Eng\\
  Comp 322\\
  12/12/13
}

\renewcommand{\headrulewidth}{0pt}

\begin{document}

\begin{center}
    \section*{Homework 4}
    \subsection*{Seventh Edition}
\end{center}

\subsection*{Chapter 7}
\begin{enumerate}
\setcounter{enumi}{1}
\item In the dining philosophers problem:
\begin{itemize}
\item \textbf{Mutual Exclusion}: When a philosopher picks up a chopstick (a resource), no other philosopher may pick up the same chopstick.
\item \textbf{Hold and Wait}: When a philosopher picks up a chopstick, he/she waits for another chopstick before eating. So the philosopher is holding one resources, and waiting for another.
\item \textbf{No preemption}: In the dining philosophers problem, each person only gives up their chopsticks after eating. So each philosopher only releases his/her resources after eating is complete. No mention of preemption, such as an omniscient being to take chopsticks away from the philosophers.
\item \textbf{Circular Wait}: If each philosopher picks up one chopstick, then philosopher 1 waits for philosopher 2 to put down the chopstick, who waits for philosopher 3, who waits for 4, who waits for 5 who is waiting for the first one.
\end{itemize}
To end the deadlock any of these conditions must be eliminated.
Mutual exclusion could be eliminated by allowing a philosopher to steal the chopstick out of one of the others hand.
%Mutual exclusion cannot be eliminated because the chopsticks cannot be shared. Only one philosopher may use a chopstick at once.
This may have other problems but there would no longer be a deadlock.
Since circular wait implies hold and wait, we would only need to fix hold and wait and both of those conditions would no longer deadlock.
%To remove the hold and wait condition, ...
To remove hold and wait, it is similar to the solutions described for mutual exclusion; allow other philosophers to take chopsticks out of the others hand
To remove no preemption, we could have an omniscient philosopher which could remove chopsticks out of a philosopher's hand when he/she has been holding onto a resource too long.
\setcounter{enumi}{5}
\item If each process tries to aquire a resource, there will be one resource left over. The remaining resource can be given to any one of the process, and it will eventually finish. This situation will never deadlock.
\setcounter{enumi}{7}
\item If it is the last chopstick remaining, check if any other philosopher is currently eating, and if no one else is, wait.
\setcounter{enumi}{10}
\item
  \begin{enumerate}
  \item \begin{tabular}{l r r r r }
    & \multicolumn{4}{c}{Need}
    \\
    \cline{2-5}
          & A & B & C & D\\
    $P_0$ & 0 & 0 & 0 & 0\\
    $P_1$ & 0 & 7 & 5 & 0\\
    $P_2$ & 1 & 0 & 0 & 2\\
    $P_3$ & 0 & 0 & 2 & 0\\
    $P_4$ & 0 & 6 & 4 & 2
  \end{tabular}
  \item The system is in a safe state. One way the processes could run is: $P_0$, (Available = (1,5,3,2)), $P_3$, (Available = (1,11,6,4)), $P_1$ (Available = (2,11,6,4)), $P_4$, (Available = (2,11,7,8)). Thus, all processes are in a safe state.
  \item If a request arrives from process $P_1$ for (0,4,2,0). The request will reduce avaliable to (1,1,0,0). The system is still safe because we can run $P_0$, $P_2$, $P_3$, $P_1$, and $P_4$.
  \end{enumerate}

\end{enumerate}

\subsection*{Chapter 8}
\begin{enumerate}
\setcounter{enumi}{0}
\item \textbf{External Fragmentation}: Occurs when processes are loaded and removed from memory, and the free memory space is broken into little pieces. \textbf{Internal Fragmentation}: Memory that is internal to a partition but is not being used.
\setcounter{enumi}{3}
\item
  \begin{enumerate}
    \item Contiguous-memory allocation: Since the blocks are right next to each other, there may be need to reorganize the blocks at some point when processes are released and fragmentation occurs.
    \item Pure segmentation: If a segment cannot grow then it is possible to have to reallocated all of the memory.
    \item Pure Paging: Relocation is not required, because each part of the data can be stored separately and thus allowing the operating system to not have to relocate.
  \end{enumerate}
\setcounter{enumi}{4}
\item \begin{enumerate}
    \item External Fragmentation: Contiguous memory allocation suffers the most from external fragmentation. This is because each process is contained in a signal section of memory, so holes develop. Pure segmentation also suffers from external fragmentation. Pure paging does not suffer from external fragmentation.
    \item Internal Fragmentation: Contiguous memory does not suffer from internal fragmentation. Similarly for pure segmentation. Pure paging suffers from internal fragmentation because it allocates blocks to each process which may be bigger than a segment of a process needs.
    \item Ability to share code across processes: Contiguous memory does not allow processes to share code. Segmentation allows a process to share code. Pure paging also allows a process to share code.
  \end{enumerate}
\setcounter{enumi}{8}
\item 
  \begin{enumerate}
  \item 400 nanoseconds
  \item 250 nanoseconds
  \end{enumerate}
\setcounter{enumi}{12}
\item If a page table becomes large enough it has the same problems as regular memory. Less commonly used things could be swapped if they are paged. Also, memory could be allocated fixed-sized instead of variable-size)
\end{enumerate}

\subsection*{Chapter 9}
\begin{enumerate}
\setcounter{enumi}{11}
\item A flag could be added to implement this. The negatives to this are that there is not too much information about the page and how long it has been around. Benefits are that the implementation is simple.
\item
\item 0.5 milliseconds
\item Thrashing is caused by continuously page faulting. This occurs when there is an underallocation of pages.
\end{enumerate}

\end{document}
